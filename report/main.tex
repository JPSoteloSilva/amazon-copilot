\documentclass[12pt]{article}

\usepackage{url}
\usepackage{geometry}
\geometry{a4paper}
\usepackage{microtype}

\usepackage{graphicx}
\usepackage{amsmath}
\usepackage{algorithm}
\usepackage[noend]{algpseudocode}
\usepackage{float}
\usepackage{wrapfig}
\usepackage{listings}
\usepackage{multirow}

\usepackage{enumitem}

%para escribir en español
\usepackage[utf8x]{inputenc}
\usepackage[spanish]{babel}

\renewcommand{\contentsname}{Índice}
\renewcommand{\refname}{Referencias}
\renewcommand{\figurename}{Figura}

\usepackage[nottoc,numbib]{tocbibind}

\linespread{1.2}

\graphicspath{{./figures/}} % Specifies the directory where pictures are stored

\begin{document}

\begin{titlepage}

    \center

    \textsc{\LARGE Universidad de la República}\\[1.5cm]
    \textsc{\Large Facultad de Ingeniería}\\[1.0cm]
    \textsc{\large Recuperación de Información y Recomendaciones en la Web}\\[0.5cm]

    \rule{\linewidth}{0.5mm} \\[1cm]
    {\huge \bfseries Amazon Copilot}\\[0.5cm]
    \rule{\linewidth}{0.5mm} \\[1.5cm]

    \begin{minipage}{0.4\textwidth}
        \begin{flushleft} \large
            \emph{Autores:}\\
            Juan Pablo Conde\\
            Xavier Iribarnegaray\\
            Juan Pablo Sotelo\\
        \end{flushleft}
    \end{minipage}
    ~
    \begin{minipage}{0.4\textwidth}
        \begin{flushright} \large
            \emph{Profesores:} \\
            Libertad Tansini \\
        \end{flushright}
    \end{minipage}\\[2cm]

    {\large \today}\\[1.5cm]

    \includegraphics[width=0.4\textwidth]{fing-logo}\\[1cm]

    \vfill

\end{titlepage}


\pagenumbering{arabic}

\tableofcontents

\newpage

\section{Introducción}
En este documento presentamos la propuesta para Amazon Copilot, un proyecto que buscará facilitar la búsqueda y selección de productos en plataformas de e-commerce. Nuestro sistema planea utilizar un dataset de productos de Amazon para implementar tres funcionalidades clave:

\begin{itemize}
    \item \textbf{Búsqueda híbrida de productos:} Combinaremos búsqueda semántica y técnicas de text matching tradicionales para ofrecer resultados más relevantes a las consultas de los usuarios.

    \item \textbf{AI Agent:} Un asistente conversacional que utilizará las capacidades
          de búsqueda implementadas y podrá refinar los requerimientos del usuario mediante
          diálogo natural.

    \item \textbf{Sistema de recomendación:} Ofrecerá sugerencias de productos relacionados con los ítems seleccionados en el carrito de compras.
\end{itemize}

El proyecto integrará técnicas modernas de procesamiento de lenguaje natural, recuperación de información y sistemas de recomendación para crear una experiencia de usuario fluida y personalizada.

\subsection{Marco Teórico}

\subsubsection{Conceptos del curso utilizados:}
Lorem ipsum dolor sit amet, consectetur adipiscing elit. Nullam in dui mauris. Vivamus hendrerit arcu sed erat molestie vehicula. Sed auctor neque eu tellus rhoncus ut eleifend nibh porttitor. Ut in nulla enim. Phasellus molestie magna non est bibendum non venenatis nisl tempor. Suspendisse dictum feugiat nisl ut dapibus. Mauris iaculis porttitor posuere.

\subsubsection{Conceptos fuera del curso utilizados:}
Cras malesuada ultrices augue molestie risus. Vestibulum congue semper purus, nec ornare nulla faucibus eget. Curabitur vitae congue leo. Pellentesque eget dictum risus. Sed consequat sollicitudin nunc, sed ultricies eros suscipit sit amet. 
\newpage
\section{Arquitectura del Sistema}

\section{Funcionalidades}

\section{Implementación Técnica}

% \section{Testing}

\section{Referencias}

\end{document}
