\documentclass[12pt]{article}

\usepackage{url}
\usepackage{geometry}
\geometry{a4paper}
\usepackage{microtype}

\usepackage{graphicx}
\usepackage{amsmath}
\usepackage{algorithm}
\usepackage[noend]{algpseudocode}
\usepackage{float}
\usepackage{wrapfig}
\usepackage{listings}
\usepackage{multirow}
\usepackage{hyperref}

\usepackage{enumitem}

%para escribir en español
\usepackage[utf8x]{inputenc}
\usepackage[spanish]{babel}

\renewcommand{\contentsname}{Índice}
\renewcommand{\refname}{Referencias}
\renewcommand{\figurename}{Figura}

\usepackage[nottoc,numbib]{tocbibind}

\linespread{1.2}

\graphicspath{{./figures/}} % Specifies the directory where pictures are stored

\begin{document}

\begin{titlepage}

    \center

    \textsc{\LARGE Universidad de la República}\\[1.5cm]
    \textsc{\Large Facultad de Ingeniería}\\[1.0cm]
    \textsc{\large Recuperación de Información y Recomendaciones en la Web}\\[0.5cm]

    \rule{\linewidth}{0.5mm} \\[1cm]
    {\huge \bfseries Amazon Copilot}\\[0.5cm]
    \rule{\linewidth}{0.5mm} \\[1.5cm]

    \begin{minipage}{0.4\textwidth}
        \begin{flushleft} \large
            \emph{Autores:}\\
            Juan Pablo Conde\\
            Xavier Iribarnegaray\\
            Juan Pablo Sotelo\\
        \end{flushleft}
    \end{minipage}
    ~
    \begin{minipage}{0.4\textwidth}
        \begin{flushright} \large
            \emph{Profesores:} \\
            Libertad Tansini \\
        \end{flushright}
    \end{minipage}\\[2cm]

    {\large \today}\\[1.5cm]

    \includegraphics[width=0.4\textwidth]{fing-logo}\\[1cm]

    \vfill

\end{titlepage}

\pagenumbering{arabic}

\tableofcontents

\newpage

\section{Introducción}
Amazon Copilot constituye un sistema diseñado para optimizar la búsqueda y selección de productos en plataformas de comercio electrónico. Basado en un conjunto de datos de productos de Amazon, el sistema implementará técnicas avanzadas de recuperación de información y procesamiento de lenguaje natural para proporcionar una experiencia de compra intuitiva y personalizada.

El proyecto integra tres funcionalidades principales: un sistema de búsqueda híbrido que combina técnicas semánticas y tradicionales para ofrecer resultados más relevantes; un asistente conversacional (Agente de Inteligencia Artificial) que proporciona una interfaz de búsqueda interactiva, refinando requerimientos mediante diálogo natural y utilizando como mecanismo subyacente el sistema de búsqueda híbrido implementado; y un sistema de recomendación que, basándose en los productos seleccionados en el carrito, emplea el mismo agente para identificar artículos similares o complementarios, mejorando así la experiencia de descubrimiento de productos.

\section{Funcionalidades del Sistema}

En esta sección se presentan las tres principales funcionalidades que se implementarán en Amazon Copilot.

\subsection{Búsqueda Híbrida de Productos}

Este componente combina técnicas de búsqueda semántica con métodos tradicionales de correspondencia textual para optimizar resultados. Incorpora representaciones vectoriales semánticas (embeddings) para capturar el significado contextual de las consultas, mantiene búsqueda por términos exactos para consultas específicas e implementa filtrado por categorías. El sistema ordena los resultados mediante un modelo que pondera tanto la relevancia semántica como la coincidencia sintáctica tradicional, logrando un equilibrio óptimo que aprovecha las fortalezas de ambos enfoques para mostrar los productos más pertinentes al usuario.

\subsection{Búsqueda Conversacional con Agente de IA}

El asistente conversacional proporciona una interfaz de búsqueda interactiva en lenguaje natural. Este agente mantiene el contexto durante la conversación, refina progresivamente los requerimientos del usuario mediante preguntas específicas, y utiliza la API de búsqueda híbrida como herramienta subyacente para ofrecer resultados personalizados. Resulta especialmente útil para usuarios sin conocimientos específicos sobre los productos que buscan.

\subsection{Sistema de Recomendación con Agente de IA}

Este sistema analiza los productos seleccionados en el carrito para sugerir artículos complementarios o alternativos. Opera automáticamente al visualizar el carrito, generando recomendaciones basadas en productos frecuentemente adquiridos conjuntamente y alternativas similares. Presenta explicaciones concisas sobre cada recomendación y adapta las sugerencias dinámicamente según el contenido del carrito y el historial de la conversación con el usuario.

\section{Arquitectura del Sistema}

Esta sección describe los componentes principales del sistema, sus responsabilidades e interacciones, así como las tecnologías propuestas para su implementación.

\vspace{1cm}

\begin{figure}[H]
    \centering
    \includegraphics[width=0.8\textwidth]{architecture.png}
    \caption{Diagrama de alto nivel de la arquitectura del sistema.}
    \label{fig:system_architecture}
\end{figure}

\subsection{API}

Este componente constituye la capa de servicios que comunica el frontend con los distintos motores de búsqueda y recomendación. Incluye endpoints para listar productos (recibiendo parámetros de filtros y búsqueda) y para obtener información detallada de productos específicos.

\subsubsection{Implementación}

La API será implementada utilizando \href{https://fastapi.tiangolo.com/}{FastAPI}~\cite{FastAPI}, un framework moderno de Python que permite un desarrollo eficiente y robusto. FastAPI ofrece validación automática de datos mediante \href{https://docs.pydantic.dev/latest/}{Pydantic}~\cite{Pydantic}, tipado estático que reduce errores en tiempo de desarrollo, y generación automática de documentación con OpenAPI y Swagger UI. Esta documentación servirá como recurso valioso para la fase de desarrollo del frontend.

\subsection{Dataset}

Para la información de productos, se utilizará el conjunto de datos de Amazon disponible en \href{https://www.kaggle.com/datasets/lokeshparab/amazon-products-dataset/data?select=Amazon-Products.csv}{Kaggle}~\cite{Amazon}. Este conjunto de datos ofrece una amplia variedad de atributos por producto, incluyendo:

\begin{itemize}
    \item Título y descripción detallada
    \item Categorías y subcategorías
    \item Precio y disponibilidad
    \item Valoraciones y número de reseñas
    \item Imágenes de productos
    \item Especificaciones técnicas
\end{itemize}

La riqueza de estos atributos permitirá implementar las modalidades de búsqueda previamente mencionadas.

\section{Search}

Esta sección describe la implementación técnica del sistema de búsqueda híbrida, desde el procesamiento inicial de datos hasta la ejecución de consultas complejas con filtros y paginación.

\subsection{Procesamiento de Datos Inicial}

El sistema implementa un pipeline robusto de procesamiento de datos que transforma el conjunto de datos original de Amazon en un formato optimizado para búsqueda vectorial. El dataset inicial contiene aproximadamente 550,000 productos, muchos de los cuales presentaban inconsistencias en los datos, imágenes no funcionales o información que requería normalización y limpieza antes de poder ser utilizada efectivamente.

\subsubsection{Limpieza y Normalización}

El proceso de limpieza se ejecuta a través de la función \texttt{clean\_data} en el módulo \texttt{utils.py}, implementando las siguientes transformaciones:

\begin{itemize}
    \item \textbf{Eliminación de valores nulos}: Se descartan productos sin campos esenciales (id, imagen, nombre, categoría, precios)
    \item \textbf{Filtrado de datos inconsistentes}: Se eliminan registros con valoraciones inválidas como \texttt{Get}, \texttt{FREE} o que contengan símbolos no numéricos
    \item \textbf{Normalización de valoraciones}: Conversión de conteos de ratings de formato string con comas a enteros
    \item \textbf{Conversión monetaria}: Transformación automática de precios en rupias indias (INR) a dólares estadounidenses usando tasa de cambio fija
    \item \textbf{Validación de URLs de imágenes}: Verificación concurrente de la accesibilidad de las imágenes de productos
\end{itemize}

\subsection{Base de Datos Vectorial: Qdrant}

La elección de Qdrant como base de datos vectorial se fundamenta en sus capacidades avanzadas para búsqueda híbrida y su arquitectura optimizada para aplicaciones de recuperación de información.

\subsubsection{Justificación Técnica}

Qdrant ofrece ventajas específicas que fueron determinantes para la implementación:

\begin{itemize}
    \item \textbf{Búsqueda híbrida nativa}: Soporte integrado para combinar embeddings densos y esparsos en una sola consulta
    \item \textbf{Modelos de embeddings integrados}: Utilización de FastEmbed para generar representaciones vectoriales sin dependencias externas
    \item \textbf{Filtrado avanzado}: Capacidad de aplicar filtros complejos por categorías, rangos de precios y otros atributos sin impacto significativo en el rendimiento
    \item \textbf{Fácil de usar}: Interfaz intuitiva y documentación clara que facilita la implementación y mantenimiento.
\end{itemize}

\subsubsection{Configuración de la Colección}

La colección se configura con parámetros específicos para optimizar el rendimiento de búsqueda:

\begin{itemize}
    \item \textbf{Vectores densos}: Dimensión 384 con distancia coseno para capturar similitud semántica
    \item \textbf{Vectores esparsos}: Implementación BM25 para coincidencias textuales exactas
    \item \textbf{Índices de payload}: Indexación automática de campos categóricos para filtrado eficiente
\end{itemize}

\subsection{Modelos de Embeddings}

El sistema implementa una arquitectura dual de embeddings que combina representaciones densas y esparsas para maximizar la calidad de los resultados de búsqueda.

\subsubsection{Embeddings Densos: Sentence-Transformers}

Los embeddings densos utilizan el modelo \texttt{sentence-transformers/all-MiniLM-L6-v2}, seleccionado por su balance óptimo entre calidad y eficiencia:

\begin{itemize}
    \item \textbf{Dimensionalidad}: 384 dimensiones que capturan representaciones semánticas ricas con un balance entre precisión, eficiencia y tamaño de los vectores.
    \item \textbf{Entrenamiento}: Pre-entrenado en grandes corpus multilingües.
    \item \textbf{Ventajas}: Excelente para capturar similitudes conceptuales, sinónimos y relaciones semánticas.
\end{itemize}

El proceso de generación de embeddings densos procesa tanto el título como la descripción del producto, creando una representación vectorial que captura el significado semántico completo del artículo.

\subsubsection{Embeddings Esparsos: BM25}

Los embeddings esparsos implementan el algoritmo BM25 a través del modelo \texttt{Qdrant/bm25}, creando representaciones vectoriales donde cada dimensión corresponde a un token específico del diccionario de la colección.

\subsubsection{Estructura del Vector Esparso}

Un embedding esparso se representa como un conjunto de pares (índice, valor) donde:

\begin{itemize}
    \item \textbf{Índice}: Posición del token en el diccionario global de Qdrant para la colección
    \item \textbf{Valor}: Score BM25 que combina la frecuencia del término (TF) con la frecuencia inversa de documento (IDF)
    \item \textbf{Esparsidad}: Solo se almacenan dimensiones con valores no-cero, optimizando el espacio de almacenamiento
\end{itemize}

\subsubsection{Cálculo de Relevancia}

El score BM25 para cada token se calcula mediante la fórmula:

\begin{itemize}
    \item \textbf{Componente TF}: Frecuencia del término normalizada por la longitud del documento usando parámetros k1=1.2 y b=0.75
    \item \textbf{Componente IDF}: Calculado a nivel de colección basado en la frecuencia del término en todos los documentos
    \item \textbf{Actualización dinámica}: Los valores IDF se recalculan automáticamente cuando se añaden nuevos productos a la colección
\end{itemize}

Esta implementación permite:

\begin{itemize}
    \item \textbf{Coincidencias exactas}: Identificación precisa de términos específicos en títulos y descripciones
    \item \textbf{Relevancia estadística}: Ponderación basada en la rareza del término en la colección completa
    \item \textbf{Robustez}: Resistencia a variaciones en la formulación de consultas
    \item \textbf{Eficiencia}: Almacenamiento optimizado que solo mantiene tokens relevantes
\end{itemize}

\subsection{Búsqueda Híbrida}

La implementación de búsqueda híbrida constituye el núcleo del sistema, combinando las fortalezas de ambos tipos de embeddings para proporcionar resultados superiores.

\subsubsection{Principio de Funcionamiento}

La búsqueda híbrida opera ejecutando simultáneamente dos consultas independientes sobre la misma colección de productos:

\begin{itemize}
    \item \textbf{Consulta densa}: Utiliza el embedding denso de la query para encontrar productos semánticamente similares mediante búsqueda por similitud coseno
    \item \textbf{Consulta esparsa}: Aplica el embedding esparso BM25 de la query para identificar productos con coincidencias textuales exactas
    \item \textbf{Ejecución paralela}: Ambas consultas se procesan simultáneamente en Qdrant, optimizando el tiempo de respuesta
    \item \textbf{Aplicación de filtros}: Los filtros de categoría y precio se aplican de manera idéntica a ambas consultas
\end{itemize}

\subsubsection{Ventajas de la Aproximación Dual}

Esta estrategia dual permite capturar diferentes aspectos de la relevancia:

\begin{itemize}
    \item \textbf{Cobertura semántica}: Los embeddings densos identifican productos conceptualmente relacionados aunque no compartan términos exactos
    \item \textbf{Precisión léxica}: Los embeddings esparsos garantizan que productos con términos clave específicos aparezcan en los resultados
    \item \textbf{Compensación mutua}: Cuando una aproximación falla, la otra puede proporcionar resultados relevantes
    \item \textbf{Robustez ante consultas}: El sistema funciona eficazmente tanto para búsquedas conceptuales como específicas
\end{itemize}

\subsubsection{Algoritmo de Fusión}

El sistema implementa una estrategia de fusión que opera en dos niveles:

\begin{itemize}
    \item \textbf{Nivel de puntuación}: Combinación ponderada de scores densos y esparsos usando Reciprocal Rank Fusion (RRF)
    \item \textbf{Normalización}: Ajuste de escalas entre diferentes tipos de scores para comparación equitativa
\end{itemize}

\subsection{Sistema de Filtrado}

El sistema implementa filtros que se aplican tanto a las consultas densas como esparsas durante la búsqueda híbrida.

\subsubsection{Filtros Disponibles}

\begin{itemize}
    \item \textbf{Categoría principal}: Filtrado por categorías como Electronics, Clothing, Home, etc.
    \item \textbf{Subcategoría}: Filtrado de segundo nivel (requiere categoría principal definida)
    \item \textbf{Rango de precios}: Filtros por precio mínimo y/o máximo en dólares
\end{itemize}

\subsubsection{Implementación}

Los filtros utilizan los índices automáticos de Qdrant sobre los campos de payload, aplicándose mediante operadores lógicos AND para combinar múltiples condiciones de filtrado.

\subsection{Paginación}

La implementación de paginación está optimizada para mantener rendimiento consistente independientemente del tamaño del conjunto de resultados.

\subsubsection{Implementación}

El sistema utiliza paginación basada en offset con optimizaciones específicas:

\begin{itemize}
    \item \textbf{Offset y limit}: Parámetros estándar para control granular de resultados
    \item \textbf{Validación de parámetros}: Verificación de valores positivos y rangos válidos
    \item \textbf{Metadatos de paginación}: Información adicional sobre total de resultados y páginas disponibles
\end{itemize}

\subsection{Interfaz de Línea de Comandos}

La CLI proporciona herramientas para gestión de datos y testing del sistema, implementada con Typer y Rich para hacer uso del sistema de búsqueda desde la terminal.

\subsubsection{Comandos Principales}

\begin{itemize}
    \item \textbf{create-collection}: Creación de colecciones con configuración automática
    \item \textbf{load-products}: Carga masiva de productos desde CSV
    \item \textbf{search-products}: Búsqueda híbrida con filtros y paginación
    \item \textbf{delete-collection}: Eliminación de colecciones
    \item \textbf{test-connection}: Verificación de conectividad con Qdrant
\end{itemize}

Esta implementación integral del sistema de búsqueda proporciona una base sólida para operaciones de recuperación de información eficientes y escalables, combinando técnicas modernas de NLP con optimizaciones específicas para comercio electrónico.

\section{Agent}

Esta sección describe la implementación técnica del agente conversacional de IA, desde la arquitectura del grafo de estados hasta la gestión de preferencias y presentación de productos.

\subsection{Arquitectura del Grafo de Estados}

El agente conversacional está implementado utilizando LangGraph, que proporciona una arquitectura basada en grafos de estados para manejar conversaciones complejas con múltiples etapas de procesamiento.

\subsubsection{Estado del Grafo}

El sistema mantiene un estado global (\texttt{GraphState}) que persiste a lo largo de toda la conversación:

\begin{itemize}
    \item \textbf{Historial de conversación}: Lista de mensajes intercambiados entre usuario y asistente
    \item \textbf{Preferencias del usuario}: Estructura que almacena criterios de búsqueda recolectados dinámicamente
    \item \textbf{Productos encontrados}: Lista de productos resultantes de la búsqueda híbrida
\end{itemize}

\subsubsection{Nodos del Grafo}

La arquitectura implementa tres nodos principales que procesan diferentes aspectos de la conversación:

\begin{itemize}
    \item \textbf{collect\_preferences}: Recolecta y refina las preferencias del usuario mediante diálogo natural
    \item \textbf{search\_products}: Ejecuta búsquedas híbridas utilizando las preferencias recolectadas
    \item \textbf{present\_products}: Genera presentaciones personalizadas de los productos encontrados
\end{itemize}

\subsection{Recolección de Preferencias}

El sistema implementa un mecanismo sofisticado para extraer preferencias del usuario a través de conversación natural, utilizando técnicas de procesamiento de lenguaje natural.

\subsubsection{Estructura de Preferencias}

Las preferencias del usuario se modelan mediante la clase \texttt{UserPreferences}:

\begin{itemize}
    \item \textbf{Query}: Consulta textual que describe el tipo de producto deseado
    \item \textbf{Categoría principal}: Categoría de alto nivel (Electronics, Clothing, Home, etc.)
    \item \textbf{Rango de precios}: Límites mínimo y máximo en dólares estadounidenses
    \item \textbf{Atributos específicos}: Color y marca preferidos por el usuario
\end{itemize}

\subsubsection{Validación de Suficiencia}

El sistema implementa una función de validación (\texttt{has\_sufficient\_preferences}) que determina cuándo se han recolectado suficientes datos para ejecutar una búsqueda efectiva:

\begin{itemize}
    \item \textbf{Campos mínimos}: Requiere al menos 3 campos completados de las 6 opciones disponibles
    \item \textbf{Campo obligatorio}: La query textual es siempre requerida para iniciar búsquedas
    \item \textbf{Validación dinámica}: Evalúa la completitud en cada iteración del diálogo
\end{itemize}

\subsubsection{Prompts Especializados}

La recolección utiliza prompts ingeniería específicamente diseñados para extraer información estructurada:

\begin{itemize}
    \item \textbf{Manejo multiidioma}: Respuestas en el idioma del usuario, datos almacenados en inglés
    \item \textbf{Inferencia categórica}: Mapeo automático de descripciones a categorías válidas
    \item \textbf{Preservación de contexto}: Mantiene preferencias previamente recolectadas
    \item \textbf{Resolución de conflictos}: Solicita clarificación ante información contradictoria
\end{itemize}

\subsubsection{Limitaciones del Sistema Multiidioma}

A pesar de las especificaciones detalladas en los prompts, el sistema presenta limitaciones en el manejo multiidioma:

\begin{itemize}
    \item \textbf{Inconsistencia en traducción}: La recolección de preferencias no siempre almacena los datos en inglés como se especifica en el prompt, lo que puede causar desajustes con la base de datos de productos
    \item \textbf{Limitaciones del modelo}: El modelo GPT-4.1-nano actual no siempre sigue consistentemente las instrucciones de traducción automática
    \item \textbf{Posible solución}: Un modelo más expresivo y con mayor capacidad de seguimiento de instrucciones podría resolver estos problemas de consistencia idiomática
\end{itemize}

\subsection{Integración con Búsqueda}

El agente utiliza el sistema de búsqueda híbrida como herramienta subyacente para obtener productos relevantes basados en las preferencias recolectadas.

\subsubsection{Construcción de Consultas}

El nodo \texttt{search\_products} construye consultas optimizadas combinando múltiples campos de preferencias:

\begin{itemize}
    \item \textbf{Query base}: Utiliza la consulta textual principal del usuario
    \item \textbf{Enriquecimiento}: Añade color y marca cuando están disponibles
    \item \textbf{Filtros aplicados}: Categoría principal y rangos de precio se aplican como filtros
    \item \textbf{Límite de resultados}: Configurado para presentar 3 productos por defecto
\end{itemize}

\subsubsection{Manejo de Errores}

El sistema implementa estrategias robustas para manejar casos donde la búsqueda no produce resultados:

\begin{itemize}
    \item \textbf{Validación previa}: Verifica que existe una query antes de ejecutar búsquedas
    \item \textbf{Respuestas alternativas}: Proporciona mensajes informativos cuando no hay productos
    \item \textbf{Recuperación graceful}: Mantiene el flujo conversacional ante errores técnicos
\end{itemize}

\subsection{Presentación de Productos}

La presentación de productos utiliza un sistema de generación de texto especializado que adapta el contenido según las preferencias del usuario y características de los productos encontrados.

\subsubsection{Generación Contextual}

El nodo \texttt{present\_products} genera descripciones personalizadas utilizando:

\begin{itemize}
    \item \textbf{Contexto de preferencias}: Incluye los criterios específicos del usuario
    \item \textbf{Datos de productos}: Información detallada de cada artículo encontrado
    \item \textbf{Orden preservado}: Mantiene la secuencia de relevancia de la búsqueda
    \item \textbf{Alineación con necesidades}: Explica cómo cada producto satisface los criterios
\end{itemize}

\subsubsection{Formato de Presentación}

Las presentaciones siguen una estructura consistente que optimiza la experiencia del usuario:

\begin{itemize}
    \item \textbf{Introducción general}: Mensaje amigable que resume las preferencias identificadas
    \item \textbf{Descripciones individuales}: Análisis detallado de cada producto con características clave
    \item \textbf{Información de precios}: Precios actuales y descuentos cuando aplican
    \item \textbf{Valoraciones sociales}: Ratings y número de reseñas para validación social
\end{itemize}

\subsection{Configuración y Parámetros}

El sistema utiliza un conjunto de parámetros configurables que optimizan el comportamiento del agente para diferentes escenarios de uso.

\subsubsection{Configuración de OpenAI}

\begin{itemize}
    \item \textbf{Modelo}: GPT-4.1-nano para balance entre calidad y eficiencia
    \item \textbf{Temperatura}: 0.0 para respuestas determinísticas y consistentes
    \item \textbf{Structured outputs}: Utiliza Pydantic para validación automática de respuestas
\end{itemize}

\subsubsection{Parámetros de Conversación}

\begin{itemize}
    \item \textbf{Contexto de mensajes}: Mantiene los últimos 10 mensajes para contexto relevante
    \item \textbf{Límite de recursión}: Máximo 10 iteraciones para prevenir bucles infinitos
    \item \textbf{Gestión de hilos}: Identificador único para mantener contexto conversacional
\end{itemize}

\subsection{Continuidad Conversacional y Refinamiento}

El sistema mantiene el contexto completo de la conversación, permitiendo refinamientos iterativos de búsquedas basados en interacciones previas.

\subsubsection{Mantenimiento de Contexto}

Una vez que se ha realizado una búsqueda inicial y se han presentado productos al usuario, el sistema permite:

\begin{itemize}
    \item \textbf{Refinamiento de preferencias}: El usuario puede modificar criterios específicos (como rango de precios) manteniendo el resto de preferencias previamente establecidas
    \item \textbf{Búsquedas iterativas}: Nuevas consultas utilizan el contexto previo para generar resultados más refinados
    \item \textbf{Preservación selectiva}: Solo se actualizan los campos de preferencias que el usuario modifica explícitamente
    \item \textbf{Historial persistente}: El sistema mantiene el registro completo de la conversación para referencias futuras
\end{itemize}

\subsubsection{Ejemplo de Refinamiento}

El flujo típico de refinamiento funciona de la siguiente manera:

\begin{itemize}
    \item \textbf{Búsqueda inicial}: Usuario solicita "auriculares inalámbricos" con presupuesto de \$50-100
    \item \textbf{Presentación de resultados}: Sistema muestra 3 productos relevantes
    \item \textbf{Refinamiento del usuario}: "Muéstrame opciones más baratas, hasta \$50"
    \item \textbf{Nueva búsqueda contextual}: Sistema mantiene "auriculares inalámbricos" pero actualiza price\_max a \$50
    \item \textbf{Resultados refinados}: Nuevos productos que satisfacen los criterios actualizados
\end{itemize}

Esta capacidad de refinamiento iterativo mejora significativamente la experiencia del usuario al permitir ajustes precisos sin necesidad de reiniciar la conversación completa.

\subsection{Flujo de Ejecución}

El agente implementa un flujo de ejecución determinístico que garantiza una experiencia conversacional coherente y eficiente.

\subsubsection{Enrutamiento Condicional}

La función \texttt{route\_after\_collection} implementa lógica de decisión que determina el siguiente paso basado en el estado de las preferencias:

\begin{itemize}
    \item \textbf{Preferencias insuficientes}: Continúa en modo recolección para obtener más información
    \item \textbf{Preferencias suficientes}: Procede a ejecutar búsqueda de productos
    \item \textbf{Evaluación dinámica}: Reevalúa en cada iteración del diálogo
\end{itemize}

\subsubsection{Manejo de Excepciones}

El sistema implementa captura robusta de errores que mantiene la funcionalidad ante fallos:

\begin{itemize}
    \item \textbf{Errores de API}: Manejo graceful de fallos en llamadas a OpenAI
    \item \textbf{Errores de búsqueda}: Recuperación ante problemas con el motor de búsqueda
    \item \textbf{Mensajes informativos}: Comunicación clara de problemas al usuario
    \item \textbf{Continuidad conversacional}: Mantiene el contexto ante errores técnicos
\end{itemize}

Esta implementación integral del agente conversacional proporciona una experiencia de usuario natural y eficiente, combinando técnicas avanzadas de procesamiento de lenguaje natural con integración robusta al sistema de búsqueda híbrida.

\section{Recommendation}

\subsection{Estructura (flujo)}

% TODO: Agregar diagrama de flujo del sistema de recomendación

\subsection{Funcionamiento}

El sistema de recomendación analiza los productos seleccionados en el carrito para sugerir artículos complementarios o alternativos. Opera automáticamente al visualizar el carrito, generando recomendaciones basadas en productos frecuentemente adquiridos conjuntamente y alternativas similares.

El sistema presenta explicaciones concisas sobre cada recomendación y adapta las sugerencias dinámicamente según el contenido del carrito y el historial de la conversación con el usuario. Utiliza el mismo agente de IA para proporcionar recomendaciones contextuales y personalizadas.

\section{UX/UI}

Este módulo define la interfaz gráfica y las interacciones del usuario con el sistema (UI/UX), constituyendo el frontend de la aplicación. A través de esta capa, los usuarios pueden acceder a todas las funcionalidades del sistema de manera intuitiva y eficiente.

\subsection{Diseño/Navegabilidad}

\begin{figure}[H]
    \centering
    \includegraphics[width=0.8\textwidth]{ux.png}
    \caption{Diseño UX de la interfaz web.}
    \label{fig:ux_design}
\end{figure}

\noindent \textit{Nota: Este diagrama modela las interacciones del usuario con el sistema. El diseño final de la interfaz de usuario está pendiente de desarrollo.}

\vspace{0.75cm}

La interfaz permite al usuario: visualizar un listado inicial de productos con barra de búsqueda y filtros por categorías; realizar búsquedas que actualizan dinámicamente los resultados; acceder a detalles de productos específicos; gestionar un carrito de compras que muestra recomendaciones relacionadas; e interactuar con el asistente conversacional para búsquedas conversacionales, manteniendo el historial de la conversación.

\subsection{Filtros, Pagination}

El sistema implementará filtros avanzados que permitan a los usuarios refinar sus búsquedas por categoría, rango de precios, valoraciones y otros atributos relevantes. La paginación eficiente garantizará una experiencia fluida incluso con grandes volúmenes de resultados.

\subsection{Estado (Cart \& Conversation)}

La gestión del estado incluye el mantenimiento persistente del carrito de compras y el historial de conversaciones con el agente de IA. Esto permite una experiencia coherente y personalizada a lo largo de toda la sesión del usuario.

\section{Expansión}

El presente capítulo traza una hoja de ruta para la evolución de Amazon Copilot, agrupando las iniciativas en tres grandes ejes: experiencia de usuario, calidad de las recomendaciones y robustez de la plataforma. Todas las ideas aquí expuestas son complementarias entre sí y pueden abordarse de forma incremental.

\subsection{Streaming de respuestas}

Implementar \textit{streaming} permitiría que el asistente enviase la respuesta de forma incremental—token a token—mediante \textit{Server-Sent Events} o \textit{WebSockets}.
Esto traería dos ventajas principales:

\begin{itemize}
    \item \textbf{Percepción de inmediatez.} El usuario ve cómo el mensaje se «escribe» en pantalla, reduciendo la sensación de espera en consultas largas.
    \item \textbf{Interrupción inteligente.} Al recibir tokens en tiempo real, el cliente puede ofrecer al usuario la opción de cancelar, reformular o profundizar sin tener que esperar al final de la generación.
\end{itemize}

\subsection{Base de datos de telemetría y \textit{feedback}}

Registrar la interacción de los usuarios en una base de datos operacional abre la puerta a un ciclo virtuoso de mejora continua:

\begin{itemize}
    \item \textbf{Patrones de comportamiento.} Analizar clics, búsquedas y compras para detectar tendencias estacionales o hábitos de compra.
    \item \textbf{Re-entrenamiento de modelos.} Utilizar los datos recogidos para ajustar la ponderación entre embeddings densos y esparcidos, o para afinar los \emph{prompts}.
    \item \textbf{Personalización.} Mantener perfiles de preferencia (marcas, rangos de precio, colores) y aplicarlos en futuras consultas o recomendaciones.
    \item \textbf{Métricas de producto.} Medir \emph{CTR}, tasa de conversión y tiempo medio de respuesta para orientar decisiones de negocio.
\end{itemize}

\subsection{Embeddings de imágenes y búsqueda multimodal}

Extender el índice a vectores visuales dotaría al sistema de capacidades multimodales:

\begin{itemize}
    \item \textbf{Búsqueda inversa.} El usuario podría subir una foto o URL y recibir productos similares en forma, color o estilo.
    \item \textbf{Recomendaciones estéticas.} Combinar señal visual y textual para sugerir artículos que «combinen» con el carrito actual (p.\,ej. sets de ropa).
    \item \textbf{Comparación rápida.} Mostrar al usuario variaciones visuales (otros colores, modelos o diseños) sin depender solo de descripciones textuales.
\end{itemize}

\subsection{Sistemas de memoria a largo plazo}

Implementar memoria persistente permitiría al sistema mantener contexto de usuario más allá de conversaciones individuales:

\begin{itemize}
    \item \textbf{Perfiles de preferencias persistentes.} Mantener un registro histórico de categorías preferidas, marcas favoritas, rangos de precio habituales y patrones de compra del usuario.
    \item \textbf{Memoria semántica.} Almacenar conceptos y relaciones aprendidas durante interacciones previas para mejorar la comprensión contextual en futuras conversaciones.
    \item \textbf{Historial de búsquedas.} Registrar consultas exitosas y productos seleccionados para identificar patrones de comportamiento y preferencias implícitas.
    \item \textbf{Adaptación progresiva.} Ajustar automáticamente los algoritmos de recomendación basándose en el feedback acumulado y las interacciones históricas del usuario.
\end{itemize}

\subsection{Análisis de tendencias y patrones estacionales}

Incorporar inteligencia temporal permitiría al sistema anticipar y responder a cambios en la demanda:

\begin{itemize}
    \item \textbf{Detección de tendencias emergentes.} Analizar patrones de búsqueda y compra para identificar productos en auge antes de que se conviertan en tendencias masivas.
    \item \textbf{Recomendaciones estacionales.} Ajustar automáticamente las sugerencias según la época del año, festividades o eventos especiales (Black Friday, Navidad, regreso a clases).
    \item \textbf{Predicción de demanda.} Utilizar datos históricos para anticipar picos de demanda en categorías específicas y optimizar la presentación de productos.
    \item \textbf{Personalización temporal.} Adaptar las recomendaciones considerando no solo las preferencias del usuario, sino también el contexto temporal y las tendencias actuales del mercado.
\end{itemize}

\section{Conclusiones}

% TODO: Desarrollar conclusiones del proyecto


\newpage

\begin{thebibliography}{2}
    \raggedright

    \bibitem{Cursor}Cursor. (2024). \textit{Cursor: The AI-first code editor}. \url{https://www.cursor.com}

    \bibitem{Replit}Replit. (2024). \textit{Replit: The collaborative browser based IDE}. \url{https://replit.com}

    \bibitem{FastAPI}FastAPI. (2024). \textit{FastAPI: Framework web de alto rendimiento para APIs con Python}. \url{https://fastapi.tiangolo.com}

    \bibitem{Pydantic}Pydantic. (2024). \textit{Pydantic: Validación de datos para Python}. \url{https://docs.pydantic.dev/latest}

    \bibitem{LangGraph}LangGraph. (2024). \textit{LangGraph: Orchestration framework for building stateful, multi-actor applications with LLMs}. \url{https://www.langchain.com/langgraph}

    \bibitem{OpenAI}OpenAI. (2024). \textit{OpenAI: Creating safe AI that benefits humanity}. \url{https://openai.com}

    \bibitem{Qdrant}Qdrant. (2024). \textit{Qdrant: Vector Database}. \url{https://qdrant.tech/documentation/search-precision/reranking-hybrid-search}

    \bibitem{Amazon}Parab, L. (2023). \textit{Amazon Products Dataset}. Kaggle. \url{https://www.kaggle.com/datasets/lokeshparab/amazon-products-dataset/data?select=Amazon-Products.csv}

\end{thebibliography}

\end{document}
