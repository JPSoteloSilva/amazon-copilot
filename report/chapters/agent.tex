\section{Agent}

Esta sección describe la implementación técnica del agente conversacional de IA, desde la arquitectura del grafo de estados hasta la gestión de preferencias y presentación de productos.

\subsection{Arquitectura del Grafo de Estados}

El agente conversacional está implementado utilizando LangGraph, que proporciona una arquitectura basada en grafos de estados para manejar conversaciones complejas con múltiples etapas de procesamiento.

\subsubsection{Estado del Grafo}

El sistema mantiene un estado global (\texttt{GraphState}) que persiste a lo largo de toda la conversación:

\begin{itemize}
    \item \textbf{Historial de conversación}: Lista de mensajes intercambiados entre usuario y asistente
    \item \textbf{Preferencias del usuario}: Estructura que almacena criterios de búsqueda recolectados dinámicamente
    \item \textbf{Productos encontrados}: Lista de productos resultantes de la búsqueda híbrida
\end{itemize}

\subsubsection{Nodos del Grafo}

La arquitectura implementa tres nodos principales que procesan diferentes aspectos de la conversación:

\begin{itemize}
    \item \textbf{collect\_preferences}: Recolecta y refina las preferencias del usuario mediante diálogo natural
    \item \textbf{search\_products}: Ejecuta búsquedas híbridas utilizando las preferencias recolectadas
    \item \textbf{present\_products}: Genera presentaciones personalizadas de los productos encontrados
\end{itemize}

\subsection{Recolección de Preferencias}

El sistema implementa un mecanismo sofisticado para extraer preferencias del usuario a través de conversación natural, utilizando técnicas de procesamiento de lenguaje natural.

\subsubsection{Estructura de Preferencias}

Las preferencias del usuario se modelan mediante la clase \texttt{UserPreferences}:

\begin{itemize}
    \item \textbf{Query}: Consulta textual que describe el tipo de producto deseado
    \item \textbf{Categoría principal}: Categoría de alto nivel (Electronics, Clothing, Home, etc.)
    \item \textbf{Rango de precios}: Límites mínimo y máximo en dólares estadounidenses
    \item \textbf{Atributos específicos}: Color y marca preferidos por el usuario
\end{itemize}

\subsubsection{Validación de Suficiencia}

El sistema implementa una función de validación (\texttt{has\_sufficient\_preferences}) que determina cuándo se han recolectado suficientes datos para ejecutar una búsqueda efectiva:

\begin{itemize}
    \item \textbf{Campos mínimos}: Requiere al menos 3 campos completados de las 6 opciones disponibles
    \item \textbf{Campo obligatorio}: La query textual es siempre requerida para iniciar búsquedas
    \item \textbf{Validación dinámica}: Evalúa la completitud en cada iteración del diálogo
\end{itemize}

\subsubsection{Prompts Especializados}

La recolección utiliza prompts ingeniería específicamente diseñados para extraer información estructurada:

\begin{itemize}
    \item \textbf{Manejo multiidioma}: Respuestas en el idioma del usuario, datos almacenados en inglés
    \item \textbf{Inferencia categórica}: Mapeo automático de descripciones a categorías válidas
    \item \textbf{Preservación de contexto}: Mantiene preferencias previamente recolectadas
    \item \textbf{Resolución de conflictos}: Solicita clarificación ante información contradictoria
\end{itemize}

\subsubsection{Limitaciones del Sistema Multiidioma}

A pesar de las especificaciones detalladas en los prompts, el sistema presenta limitaciones en el manejo multiidioma:

\begin{itemize}
    \item \textbf{Inconsistencia en traducción}: La recolección de preferencias no siempre almacena los datos en inglés como se especifica en el prompt, lo que puede causar desajustes con la base de datos de productos
    \item \textbf{Limitaciones del modelo}: El modelo GPT-4.1-nano actual no siempre sigue consistentemente las instrucciones de traducción automática
    \item \textbf{Posible solución}: Un modelo más expresivo y con mayor capacidad de seguimiento de instrucciones podría resolver estos problemas de consistencia idiomática
\end{itemize}

\subsection{Integración con Búsqueda}

El agente utiliza el sistema de búsqueda híbrida como herramienta subyacente para obtener productos relevantes basados en las preferencias recolectadas.

\subsubsection{Construcción de Consultas}

El nodo \texttt{search\_products} construye consultas optimizadas combinando múltiples campos de preferencias:

\begin{itemize}
    \item \textbf{Query base}: Utiliza la consulta textual principal del usuario
    \item \textbf{Enriquecimiento}: Añade color y marca cuando están disponibles
    \item \textbf{Filtros aplicados}: Categoría principal y rangos de precio se aplican como filtros
    \item \textbf{Límite de resultados}: Configurado para presentar 3 productos por defecto
\end{itemize}

\subsubsection{Manejo de Errores}

El sistema implementa estrategias robustas para manejar casos donde la búsqueda no produce resultados:

\begin{itemize}
    \item \textbf{Validación previa}: Verifica que existe una query antes de ejecutar búsquedas
    \item \textbf{Respuestas alternativas}: Proporciona mensajes informativos cuando no hay productos
    \item \textbf{Recuperación graceful}: Mantiene el flujo conversacional ante errores técnicos
\end{itemize}

\subsection{Presentación de Productos}

La presentación de productos utiliza un sistema de generación de texto especializado que adapta el contenido según las preferencias del usuario y características de los productos encontrados.

\subsubsection{Generación Contextual}

El nodo \texttt{present\_products} genera descripciones personalizadas utilizando:

\begin{itemize}
    \item \textbf{Contexto de preferencias}: Incluye los criterios específicos del usuario
    \item \textbf{Datos de productos}: Información detallada de cada artículo encontrado
    \item \textbf{Orden preservado}: Mantiene la secuencia de relevancia de la búsqueda
    \item \textbf{Alineación con necesidades}: Explica cómo cada producto satisface los criterios
\end{itemize}

\subsubsection{Formato de Presentación}

Las presentaciones siguen una estructura consistente que optimiza la experiencia del usuario:

\begin{itemize}
    \item \textbf{Introducción general}: Mensaje amigable que resume las preferencias identificadas
    \item \textbf{Descripciones individuales}: Análisis detallado de cada producto con características clave
    \item \textbf{Información de precios}: Precios actuales y descuentos cuando aplican
    \item \textbf{Valoraciones sociales}: Ratings y número de reseñas para validación social
\end{itemize}

\subsection{Configuración y Parámetros}

El sistema utiliza un conjunto de parámetros configurables que optimizan el comportamiento del agente para diferentes escenarios de uso.

\subsubsection{Configuración de OpenAI}

\begin{itemize}
    \item \textbf{Modelo}: GPT-4.1-nano para balance entre calidad y eficiencia
    \item \textbf{Temperatura}: 0.0 para respuestas determinísticas y consistentes
    \item \textbf{Structured outputs}: Utiliza Pydantic para validación automática de respuestas
\end{itemize}

\subsubsection{Parámetros de Conversación}

\begin{itemize}
    \item \textbf{Contexto de mensajes}: Mantiene los últimos 10 mensajes para contexto relevante
    \item \textbf{Límite de recursión}: Máximo 10 iteraciones para prevenir bucles infinitos
    \item \textbf{Gestión de hilos}: Identificador único para mantener contexto conversacional
\end{itemize}

\subsection{Continuidad Conversacional y Refinamiento}

El sistema mantiene el contexto completo de la conversación, permitiendo refinamientos iterativos de búsquedas basados en interacciones previas.

\subsubsection{Mantenimiento de Contexto}

Una vez que se ha realizado una búsqueda inicial y se han presentado productos al usuario, el sistema permite:

\begin{itemize}
    \item \textbf{Refinamiento de preferencias}: El usuario puede modificar criterios específicos (como rango de precios) manteniendo el resto de preferencias previamente establecidas
    \item \textbf{Búsquedas iterativas}: Nuevas consultas utilizan el contexto previo para generar resultados más refinados
    \item \textbf{Preservación selectiva}: Solo se actualizan los campos de preferencias que el usuario modifica explícitamente
    \item \textbf{Historial persistente}: El sistema mantiene el registro completo de la conversación para referencias futuras
\end{itemize}

\subsubsection{Ejemplo de Refinamiento}

El flujo típico de refinamiento funciona de la siguiente manera:

\begin{itemize}
    \item \textbf{Búsqueda inicial}: Usuario solicita "auriculares inalámbricos" con presupuesto de \$50-100
    \item \textbf{Presentación de resultados}: Sistema muestra 3 productos relevantes
    \item \textbf{Refinamiento del usuario}: "Muéstrame opciones más baratas, hasta \$50"
    \item \textbf{Nueva búsqueda contextual}: Sistema mantiene "auriculares inalámbricos" pero actualiza price\_max a \$50
    \item \textbf{Resultados refinados}: Nuevos productos que satisfacen los criterios actualizados
\end{itemize}

Esta capacidad de refinamiento iterativo mejora significativamente la experiencia del usuario al permitir ajustes precisos sin necesidad de reiniciar la conversación completa.

\subsection{Flujo de Ejecución}

El agente implementa un flujo de ejecución determinístico que garantiza una experiencia conversacional coherente y eficiente.

\subsubsection{Enrutamiento Condicional}

La función \texttt{route\_after\_collection} implementa lógica de decisión que determina el siguiente paso basado en el estado de las preferencias:

\begin{itemize}
    \item \textbf{Preferencias insuficientes}: Continúa en modo recolección para obtener más información
    \item \textbf{Preferencias suficientes}: Procede a ejecutar búsqueda de productos
    \item \textbf{Evaluación dinámica}: Reevalúa en cada iteración del diálogo
\end{itemize}

\subsubsection{Manejo de Excepciones}

El sistema implementa captura robusta de errores que mantiene la funcionalidad ante fallos:

\begin{itemize}
    \item \textbf{Errores de API}: Manejo graceful de fallos en llamadas a OpenAI
    \item \textbf{Errores de búsqueda}: Recuperación ante problemas con el motor de búsqueda
    \item \textbf{Mensajes informativos}: Comunicación clara de problemas al usuario
    \item \textbf{Continuidad conversacional}: Mantiene el contexto ante errores técnicos
\end{itemize}

Esta implementación integral del agente conversacional proporciona una experiencia de usuario natural y eficiente, combinando técnicas avanzadas de procesamiento de lenguaje natural con integración robusta al sistema de búsqueda híbrida.
