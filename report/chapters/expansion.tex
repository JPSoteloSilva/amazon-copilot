\section{Expansión}

El presente capítulo traza una hoja de ruta para la evolución de Amazon Copilot, agrupando las iniciativas en tres grandes ejes: experiencia de usuario, calidad de las recomendaciones y robustez de la plataforma. Todas las ideas aquí expuestas son complementarias entre sí y pueden abordarse de forma incremental.

\subsection{Streaming de respuestas}

Implementar \textit{streaming} permitiría que el asistente enviase la respuesta de forma incremental—token a token—mediante \textit{Server-Sent Events} o \textit{WebSockets}.
Esto traería dos ventajas principales:

\begin{itemize}
    \item \textbf{Percepción de inmediatez.} El usuario ve cómo el mensaje se «escribe» en pantalla, reduciendo la sensación de espera en consultas largas.
    \item \textbf{Interrupción inteligente.} Al recibir tokens en tiempo real, el cliente puede ofrecer al usuario la opción de cancelar, reformular o profundizar sin tener que esperar al final de la generación.
\end{itemize}

\subsection{Base de datos de telemetría y \textit{feedback}}

Registrar la interacción de los usuarios en una base de datos operacional abre la puerta a un ciclo virtuoso de mejora continua:

\begin{itemize}
    \item \textbf{Patrones de comportamiento.} Analizar clics, búsquedas y compras para detectar tendencias estacionales o hábitos de compra.
    \item \textbf{Re-entrenamiento de modelos.} Utilizar los datos recogidos para ajustar la ponderación entre embeddings densos y esparcidos, o para afinar los \emph{prompts}.
    \item \textbf{Personalización.} Mantener perfiles de preferencia (marcas, rangos de precio, colores) y aplicarlos en futuras consultas o recomendaciones.
    \item \textbf{Métricas de producto.} Medir \emph{CTR}, tasa de conversión y tiempo medio de respuesta para orientar decisiones de negocio.
\end{itemize}

\subsection{Embeddings de imágenes y búsqueda multimodal}

Extender el índice a vectores visuales dotaría al sistema de capacidades multimodales:

\begin{itemize}
    \item \textbf{Búsqueda inversa.} El usuario podría subir una foto o URL y recibir productos similares en forma, color o estilo.
    \item \textbf{Recomendaciones estéticas.} Combinar señal visual y textual para sugerir artículos que «combinen» con el carrito actual (p.\,ej. sets de ropa).
    \item \textbf{Comparación rápida.} Mostrar al usuario variaciones visuales (otros colores, modelos o diseños) sin depender solo de descripciones textuales.
\end{itemize}
